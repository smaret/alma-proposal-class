%%%%%%%%%%%%%%%%%%%%%%%%%%%%%%%%%%%%%%%%%%%%%%%%%%%%%%%%%%%%%%
%% LaTeX template for the science justification to be       %%
%%     submitted as part of a regular ALMA proposal.        %%
%% This template should also be used for a ToO, Solar, or   %%
%%     mm-VLBI ALMA proposal, but NOT for Large Programs    %%
%%     (these have a separate template with more sections)  %%
%%                                                          %%
%%                      ALMA Cycle 6                        %%
%%                                                          %%
%%%%%%%%%%%%%%%%%%%%%%%%%%%%%%%%%%%%%%%%%%%%%%%%%%%%%%%%%%%%%%

%%%%%%%%%%%%%%%%%%%%%%%%%%%%%%%%%%%%%%%%%%%%%%%%%%
%%%%% How to convert this document to PDF %%%%%%%%
%%%%%%%%%%%%%%%%%%%%%%%%%%%%%%%%%%%%%%%%%%%%%%%%%%

% If your figures are stored as PostScript files, you can use the 
% following commands to generate a PDF file of your proposal:

%% latex file.tex
%% dvips file.dvi
%% ps2pdf file.ps file.pdf 


% If your figures are PDF images or bitmap pictures in PNG, JPG, or GIF format,
% you can use the pdflatex command to generate a PDF file from this template
% (note, however, that the pdflatex command does not handle PostScript files):

% pdflatex file.tex

% WARNINGS: 
%           1. You must make sure that PDF output generated from this
%              template is complete both when displayed with a viewer 
%              (acroread, for example) and when printed on paper.
%              LaTeX installations vary greatly and therefore it might 
%              not be possible to get all proposals to come out 
%              correctly with a single text page layout. 
%              In some cases you will have to adjust the 
%              \topmargin=-7mm command in the template to center the 
%              text vertically in the page.  
%           2. The scientific justification, figures, tables, references,
%              and public outreach statement must all fit within the
%              4-page limit.
%           3. You are free to include colour images in your proposal 
%              justification. Proposals are distributed to ALMA Review Panels 
%              in electronic form. However, the scientific content of the 
%              images should still remain clear when displayed or printed
%              in black and white.
%           4. this template is for regular, ToO, Solar, or mm-VLBI ALMA proposals,
%              but NOT for Large Programs: these have a separate template with
%              more sections, and is available from the ALMA Science Portal


%%%%%%%%%%%%%%%%%%%%%%%%%%%%%%%%%%%%%%%%%%%%%%
%%%%% Default format: 12pt single column %%%%%
%% 12pt is the minimum font size allowed !! %%
%% This applies to everything, including    %%
%% references, figure captions, and tables  %%
%%%%%%%%%%%%%%%%%%%%%%%%%%%%%%%%%%%%%%%%%%%%%%

\documentclass{alma_proposal}

\usepackage{graphics,graphicx}

%%%%%%%%%%%%%%%%%%%%%%%%%%%%%
%%%%% Start of document %%%%% 
%%%%%%%%%%%%%%%%%%%%%%%%%%%%%

\begin{document}
 
%% The title, abstract and list of PIs does not need to be included in the
%% Scientific justification, as this information is already on the cover page

%%%%%%%%%%%%%%%%%%%%%%%%%%%%%%%%%%%%%%%%%
%%%%% Body of science justification %%%%%
%%%%%%%%%%%%%%%%%%%%%%%%%%%%%%%%%%%%%%%%%

%% ENTER TEXT, FIGURES AND TABLES BELOW
%% Minimum font size for all text, references, figure captions, and tables is 12pt !!

\section{Scientific justification}

% Please describe the scientific background of the project,
% pertinent references and previous work relevant to this 
% proposal, together with any figures and tables that you judge necessary
% (use the following two examples as templates, or remove)
 
%-----------------------------Figure Start---------------------------

% The 'scale' parameter below allows you to scale the figure so that it fits within the page.
% In this case the figure was scaled to 20% of its original size.
% Note: for .png files one has to use pdflatex, not classic latex
% Minumum font size for captions: 12pt 

\begin{figure}[tbh]
\includegraphics[scale=0.2]{CO_velfield.png}
\caption{The CO(1-0) velocity field of NGC\,3256, with contours 
of the total line emission map overlaid (ALMA Science Verification Data).}
\end{figure}
%-----------------------------Figure End------------------------------

%-----------------------------Table Start-----------------------------

% Minumum font size for tables: 12pt 

\begin{table}[tbh]
\begin{center}
\caption[]{Here we show the continuum sensitivity required per band.}
\begin{tabular}{cc}
\hline \noalign {\smallskip}
Frequency (GHz) & Sensitivity (mJy) \\
\hline \noalign {\smallskip}
100 & 0.01 \\
300 & 0.10 \\
%\hline \noalign {\smallskip}
\end{tabular}
\end{center}
\end{table}
%-----------------------------Table End ------------------------------

\section{Description of observations}

% Please describe the observations to be made and their specific
% purpose, with a clear explanation of the need for, and 
% appropriateness of, ALMA Cycle 6 data.  


%%%%%%%%%%%%%%%%%%%%%%%%%%%%%
%% References section:     %%
%% Minumum font size: 12pt %%
%%%%%%%%%%%%%%%%%%%%%%%%%%%%%

\section{References}

% List references here

\noindent [1] Author1 et al. year, journal, vol, page

\noindent [2] Author2 et al. year, journal, vol, page


%%%%%%%%%%%%%%%%%%%%%%%%%%%
%%%%% End of document %%%%%
%%%%%%%%%%%%%%%%%%%%%%%%%%%

\end{document}

%%% Local Variables:
%%% mode: latex
%%% TeX-master: "L"
%%% End:
